\section{Ejercicio 2}

\subsection{Introducción}
Este ejercicio consistio en crear un modelo de mapeo de caracteristicas auto-organizadas con el objetivo de clasificar documentos. El mapa
auto-organizado que se utilizo fue una grilla bidimensional de 10 filas y 10 columnas. El objetivo de este ejercicio es el de observar espacialamente
en la grilla las distintas clases de los datos.

Para el entrenamiento se utilizo la siguiente formula para obtener la neurona ganadora $k^*$:
  \[
  k^* = \argmin_{i,j} \lvert\lvert x-w_{i,j} \rvert\rvert
  \]

Las funciones de enfriamiento de $\eta y \sigma$ que se utilizaron fueron las siguientes:
\[
  \begin{array}{ccc}
    \eta_t & = & \eta_0 * e^{\frac{-x}{\tau_1}} \\
    \sigma_t & = & \sigma_0 * e^{\frac{-x}{\tau_2}} \\
  \end{array}
   \\ \mbox{                     con } t = 1 \dots epochs
\]
\subsection{Resultados}
Para el entrenamiento se utilizaron los siguientes parametros:
%%%%%%%%%%%%%%%%%%%%%%%%%%%%%%%% AGREGAR PARAMETROS
\begin{center}
  \begin{tabular}{|c|c|c|c|}
    \hline
    $\eta_0$ & $\sigma_0$ & $\tau_1$ & $\tau_2$ \\
    \hline
    fruta  & fruta & fruta & fruta \\
    \hline
  \end{tabular}
\end{center}

Para la visualizacion de los resultados se realizo un grafico que por cada muestra del conjunto de entrenamiento
 con su correspondiente etiqueta, se calculo cual fue la neurona ganadora. Luego se calculo por cada neurona cual fue la etiqueta
 que mas la activo y se la coloreo en base a esa categoria.
Los resultados obtenidos fueron los siguientes
\subsection{Conclusión}
