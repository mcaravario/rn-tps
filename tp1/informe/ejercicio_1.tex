\section{Ejercicio 1}

\subsection{Introducción}
Este ejercicio consiste en entrenar una red neuronal que, dados los resultados de un examen específico que es utilizado en el diagnóstico del
cáncer de mamas, sepa clasificar ese conjunto dentro de dos categorias posibles: B y M. Para el este ejercicio B significará que el tumor diagnosticado
es benigno y M será maligno.

\subsection{Experimentación}
Al ser un ejercicio de clasificacion la red neuronal deberá devolver un valor para clase, por lo que la ultima función de activacion necesariamente
será la función signo, la cual devolverá -1 en caso de pertenecer la muestra a la clase M y 1 en caso de pertenecer a la clase B.

\subsubsection{Experimento 1}
Este experimento consistió en comparar distintas arquitecturas de redes neuronales variando la cantidad de capas ocultas y de neuronas por capa.
Para este experimento se decidió fijar el coeficiente de aprendizaje $\eta$ en 0.03, la cantidad de epocas sobre las que se entrena la red en 1000 y el
método de entrenamiento estocástico. Con respecto a las funcioes de activación se utilizaron sigmoideas para las capas intermedias y la función signo
para la capa final.

Las redes que se utilizaron fueron las siguientes:
%%%%%%%%%%%%%%% PONER DIBUJITO DE LA RED SI NOS DA EL TIEMPO %%%%%%%%%%%%%%%%%%%%%%%%%%%%
red 1: 10 - 1
red 2: 10 - 20 - 1
red 3: 10 - 5 - 5 - 8 - 1

Los resultados que se obtuvieron fueron los siguientes:
%%%%%%%%%%%%%%%% GRAFICOS EXPERIMENTO 1 %%%%%%%%%%%%%%%%%%%%%%%%%%%%%%%%%%%%%%%

%%%%%%%%%%%%%%% TABLA DE ACIERTOS %%%%%%%%%%%%%%%%%%%%%%%%%%%%%%%%%%%%%%%%%%%
 Tal como se observa en los resultados de la red 1, la cual es un perceptrón simple, el error cuadratico medio (ECM) no es una función
 Esto tambien se evidencia en la matriz de aciertos en la cual logra un 63.41\% de aciertos sobre los datos de entrenamiento y un 62.19\% sobre datos de validación,
 lo cual es bastante bajo. Con toda la evidencia provista por los graficos se concluyó que el problema a resolver no es linealmente separable pues no es posible
 aprenderlo con un perceptrón simple, por lo tanto esta red se descartó para futuros experimentos.

 Con respecto a la red 3, se observa que el ECM decrece tanto en validacion como en entrenamiento, comenzando en 350 sobre validacion y
 logrando un valor minimo de 265. En la matriz de aciertos se observa un 67.47\% de aciertos sobre entrenamiento y un 66.46\% sobre validacion.
 A diferencia de la red 1, se observa un decrecimiento en la funcion de ECM por lo que se concluyo que esta red logra aprender, pero la cantidad
 de aciertos sigue siendo baja. Se decidió continuar utilizando esta red para experimentar con el objetivo de analizar su comportamiento al optimizar
 el algoritmo de BackPropagation.

 Finalmente, la red 2 es la que mejor aprende de los datos de entrenamiento llegando al valor minimo de 0 en ECM sobre datos de entrenamiento y un valor minimo
 de 90 en ECM sobre datos de validación. Tambien se observa que la funcion del ECM es claramente decreciente por lo que se concluye que este perceptron multicapa
 logra aprender, resta comprobar con mas experimentos si se esta realizando \textit{overfitting} sobre los datos. En la matriz de aciertos se observó un alto
 porcentaje de aciertos, 100\% y 89.63\% para entrenamiento y validacion respectivamente por lo que se decidió seguir utilizando esta red para futuros experimentos
 con el objetivo de mejorar el porcentaje de aciertos al hacer mejoras al algoritmo de BackPropagation.

\subsubsection{Experimento 2}
Para este experimento se decidió variar el coeficiente de aprendizaje $\eta$ e introducir la optimizacion del \textit{momentum} al algoritmo de BackPropagation,
variando su respectivo coeficiente $\alpha$.
Las redes utilizadas fueron las redes 2 y 3 del Experimento 1, con 500 epocas y un modo entrenamiento estocastico. El valor de $\eta$ lo variamos con los
siguientes valores: 0.03 y 0.07, mientras que el valor de $\alpha$ se varió con los valores: 0.1 y 0.3, con lo cual se obtuvieron 4 combinaciones
posibles para cada red, las cuales se presentan a continuacion:
%%%%%%%%%%%%%%%% GRAFICOS EXPERIMENTO 2 %%%%%%%%%%%%%%%%%%%%%%%%%%%%%%%%%%%%%%%

%%%%%%%%%%%%%%% TABLA DE ACIERTOS %%%%%%%%%%%%%%%%%%%%%%%%%%%%%%%%%%%%%%%%%%5
Para la primera combinacion de parametros, la cual es $\eta = 0.07$ y $\alpha = 0.1$ se observó que la red 2 converge en una menor cantidad de epocas
 y que el ECM minimo sobre validación es menor en comparación al experimento anterior. %% Falta hablar de la cantidad de aciertos
 En la red 3 no se observó mejora sino que la red empeoró, tanto en cantidad de aciertos como en ECM sobre validación.

En la segunda combinacion de parametros ($\eta = 0.03$ y $\alpha = 0.1$) no se observaron mejoras sobre la red 2 con respecto a la primera combinacion
de parametros, es mas, la convergencia tarda mas epocas. A diferencia de la red 2, en la red 3 si se observaron mejoras con respecto a los parametros
anteriores, ya que ahora el ECM minimo sobre datos de validacion alcanza el valor de 205.

Con respecto a la tercera combinacion ($\eta = 0.03$ y $\alpha = 0.3$) no se observan mejoras en la red 2 ni en la red 3.

Finalmente con la cuarta combinacion($\eta = 0.07$ y $\alpha = 0.3$) no se observan mejoras en la red 3, pero en la red 2 el ECM alcanza valores por
debajo de los 200.

%% TERMINAR
A raíz de estos experimentos se pudo concluir que ... y se decidio continuar experimentando con la red 3

\subsubsection{Experimento 3}
En este experimento se decidió experimentar con parametros adaptativos y sus respectivos coeficientes $a$ y $b$. Se utilizó la red 3 definida
previamente, fijando la cantidad de epocas en 500, el modo de entrenamiento estocástico, $\eta = 0.03$ y $\alpha = 0.3$. Con respecto a los coeficientes
de los parametros adaptativos, se fijo $a = 0.02$ y $b$ se varió con los valores : 0.7 y 0.1.

Los resultados obtenidos fueron los siguientes:
%%%%%%%%%%%%%%%% GRAFICOS EXPERIMENTO 3 %%%%%%%%%%%%%%%%%%%%%%%%%%%%%%%%%%%%%%%

%%%%%%%%%%%%%%% TABLA DE ACIERTOS %%%%%%%%%%%%%%%%%%%%%%%%%%%%%%%%%%%%%%%%%%5

Para la primera combinacion de parametros($a = 0.02$ y $b = 0.7$) se observa una mejora en el grafico del ECM, ya que en este caso se alcanzaron valores
muy por debajo de los obtenidos con la mejor configuracion de parametros del experimento anterior. Esto no se vió reflejado en la tabla de aciertos, ya
que el porcentaje de aciertos es menor al obtenido en el experimento 2.

Para la segundo segunda combinacion de parametros($a = 0.02$ y $b = 0.1$) 

\subsection{Conclusión}

\newpage
